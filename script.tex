\documentclass[]{article}
\usepackage{lmodern}
\usepackage{amssymb,amsmath}
\usepackage{ifxetex,ifluatex}
\usepackage{fixltx2e} % provides \textsubscript
\ifnum 0\ifxetex 1\fi\ifluatex 1\fi=0 % if pdftex
  \usepackage[T1]{fontenc}
  \usepackage[utf8]{inputenc}
\else % if luatex or xelatex
  \ifxetex
    \usepackage{mathspec}
  \else
    \usepackage{fontspec}
  \fi
  \defaultfontfeatures{Ligatures=TeX,Scale=MatchLowercase}
\fi
% use upquote if available, for straight quotes in verbatim environments
\IfFileExists{upquote.sty}{\usepackage{upquote}}{}
% use microtype if available
\IfFileExists{microtype.sty}{%
\usepackage{microtype}
\UseMicrotypeSet[protrusion]{basicmath} % disable protrusion for tt fonts
}{}
\usepackage[margin=1in]{geometry}
\usepackage{hyperref}
\hypersetup{unicode=true,
            pdftitle={Trabalho - Inferência Bayesiana},
            pdfauthor={Docente: Renata Souza Bueno; Discente: Ewerson Carneiro Pimenta},
            pdfborder={0 0 0},
            breaklinks=true}
\urlstyle{same}  % don't use monospace font for urls
\usepackage{color}
\usepackage{fancyvrb}
\newcommand{\VerbBar}{|}
\newcommand{\VERB}{\Verb[commandchars=\\\{\}]}
\DefineVerbatimEnvironment{Highlighting}{Verbatim}{commandchars=\\\{\}}
% Add ',fontsize=\small' for more characters per line
\usepackage{framed}
\definecolor{shadecolor}{RGB}{248,248,248}
\newenvironment{Shaded}{\begin{snugshade}}{\end{snugshade}}
\newcommand{\AlertTok}[1]{\textcolor[rgb]{0.94,0.16,0.16}{#1}}
\newcommand{\AnnotationTok}[1]{\textcolor[rgb]{0.56,0.35,0.01}{\textbf{\textit{#1}}}}
\newcommand{\AttributeTok}[1]{\textcolor[rgb]{0.77,0.63,0.00}{#1}}
\newcommand{\BaseNTok}[1]{\textcolor[rgb]{0.00,0.00,0.81}{#1}}
\newcommand{\BuiltInTok}[1]{#1}
\newcommand{\CharTok}[1]{\textcolor[rgb]{0.31,0.60,0.02}{#1}}
\newcommand{\CommentTok}[1]{\textcolor[rgb]{0.56,0.35,0.01}{\textit{#1}}}
\newcommand{\CommentVarTok}[1]{\textcolor[rgb]{0.56,0.35,0.01}{\textbf{\textit{#1}}}}
\newcommand{\ConstantTok}[1]{\textcolor[rgb]{0.00,0.00,0.00}{#1}}
\newcommand{\ControlFlowTok}[1]{\textcolor[rgb]{0.13,0.29,0.53}{\textbf{#1}}}
\newcommand{\DataTypeTok}[1]{\textcolor[rgb]{0.13,0.29,0.53}{#1}}
\newcommand{\DecValTok}[1]{\textcolor[rgb]{0.00,0.00,0.81}{#1}}
\newcommand{\DocumentationTok}[1]{\textcolor[rgb]{0.56,0.35,0.01}{\textbf{\textit{#1}}}}
\newcommand{\ErrorTok}[1]{\textcolor[rgb]{0.64,0.00,0.00}{\textbf{#1}}}
\newcommand{\ExtensionTok}[1]{#1}
\newcommand{\FloatTok}[1]{\textcolor[rgb]{0.00,0.00,0.81}{#1}}
\newcommand{\FunctionTok}[1]{\textcolor[rgb]{0.00,0.00,0.00}{#1}}
\newcommand{\ImportTok}[1]{#1}
\newcommand{\InformationTok}[1]{\textcolor[rgb]{0.56,0.35,0.01}{\textbf{\textit{#1}}}}
\newcommand{\KeywordTok}[1]{\textcolor[rgb]{0.13,0.29,0.53}{\textbf{#1}}}
\newcommand{\NormalTok}[1]{#1}
\newcommand{\OperatorTok}[1]{\textcolor[rgb]{0.81,0.36,0.00}{\textbf{#1}}}
\newcommand{\OtherTok}[1]{\textcolor[rgb]{0.56,0.35,0.01}{#1}}
\newcommand{\PreprocessorTok}[1]{\textcolor[rgb]{0.56,0.35,0.01}{\textit{#1}}}
\newcommand{\RegionMarkerTok}[1]{#1}
\newcommand{\SpecialCharTok}[1]{\textcolor[rgb]{0.00,0.00,0.00}{#1}}
\newcommand{\SpecialStringTok}[1]{\textcolor[rgb]{0.31,0.60,0.02}{#1}}
\newcommand{\StringTok}[1]{\textcolor[rgb]{0.31,0.60,0.02}{#1}}
\newcommand{\VariableTok}[1]{\textcolor[rgb]{0.00,0.00,0.00}{#1}}
\newcommand{\VerbatimStringTok}[1]{\textcolor[rgb]{0.31,0.60,0.02}{#1}}
\newcommand{\WarningTok}[1]{\textcolor[rgb]{0.56,0.35,0.01}{\textbf{\textit{#1}}}}
\usepackage{graphicx}
% grffile has become a legacy package: https://ctan.org/pkg/grffile
\IfFileExists{grffile.sty}{%
\usepackage{grffile}
}{}
\makeatletter
\def\maxwidth{\ifdim\Gin@nat@width>\linewidth\linewidth\else\Gin@nat@width\fi}
\def\maxheight{\ifdim\Gin@nat@height>\textheight\textheight\else\Gin@nat@height\fi}
\makeatother
% Scale images if necessary, so that they will not overflow the page
% margins by default, and it is still possible to overwrite the defaults
% using explicit options in \includegraphics[width, height, ...]{}
\setkeys{Gin}{width=\maxwidth,height=\maxheight,keepaspectratio}
\IfFileExists{parskip.sty}{%
\usepackage{parskip}
}{% else
\setlength{\parindent}{0pt}
\setlength{\parskip}{6pt plus 2pt minus 1pt}
}
\setlength{\emergencystretch}{3em}  % prevent overfull lines
\providecommand{\tightlist}{%
  \setlength{\itemsep}{0pt}\setlength{\parskip}{0pt}}
\setcounter{secnumdepth}{0}
% Redefines (sub)paragraphs to behave more like sections
\ifx\paragraph\undefined\else
\let\oldparagraph\paragraph
\renewcommand{\paragraph}[1]{\oldparagraph{#1}\mbox{}}
\fi
\ifx\subparagraph\undefined\else
\let\oldsubparagraph\subparagraph
\renewcommand{\subparagraph}[1]{\oldsubparagraph{#1}\mbox{}}
\fi

%%% Use protect on footnotes to avoid problems with footnotes in titles
\let\rmarkdownfootnote\footnote%
\def\footnote{\protect\rmarkdownfootnote}

%%% Change title format to be more compact
\usepackage{titling}

% Create subtitle command for use in maketitle
\providecommand{\subtitle}[1]{
  \posttitle{
    \begin{center}\large#1\end{center}
    }
}

\setlength{\droptitle}{-2em}

  \title{Trabalho - Inferência Bayesiana}
    \pretitle{\vspace{\droptitle}\centering\huge}
  \posttitle{\par}
  \subtitle{Escola Nacional de Ciências Estatísticas (ENCE / IBGE)}
  \author{Docente: Renata Souza Bueno \\ Discente: Ewerson Carneiro Pimenta}
    \preauthor{\centering\large\emph}
  \postauthor{\par}
      \predate{\centering\large\emph}
  \postdate{\par}
    \date{Rio de Janeiro, 30, novembro de 2019}

\usepackage{booktabs}
\usepackage{longtable}
\usepackage{array}
\usepackage{multirow}
\usepackage{wrapfig}
\usepackage{float}
\usepackage{colortbl}
\usepackage{pdflscape}
\usepackage{tabu}
\usepackage{threeparttable}
\usepackage{threeparttablex}
\usepackage[normalem]{ulem}
\usepackage{makecell}
\usepackage{xcolor}

\begin{document}
\maketitle

\begin{quote}
Esse trabalho tem como objetivo verificar o conhecimento e aplicação das
técnicas da abordagem de inferência bayesiana, utilizando simulação
interativa através do método MCMC (com o algoritmo de
Metropolis-Hastings e o amostrador de Gibbs) para encontrar as
estimativas do parâmetros de uma distribuição a posteriori, cuja
distribuição dos dados segue distribuição t-Student.
\end{quote}

\texttt{Data\ de\ entrega:\ 07/12\ (sábado)}

\hypertarget{situauxe7uxe3o-1-distribuiuxe7uxe3o-t-student}{%
\subsection{Situação 1: Distribuição
t-Student}\label{situauxe7uxe3o-1-distribuiuxe7uxe3o-t-student}}

Uma variável aleatória \(Y\) tem distribuição t-Student se sua função de
densidade é dada por

\[
f(y|\mu,\,\sigma,\,\nu) = \frac{\Gamma(\frac{\nu+1}{2})} {\sqrt{\nu\pi}\,\Gamma(\frac{\nu}{2})} \left(1+\frac{1}{\nu}\left(\frac{y-\mu}{\sigma}\right)^{2} \right)^{\!-\frac{\nu+1}{2}},\,y\in \mathbb{R}\!
\]

em que \(\mu\in\mathbb{R}\) é um parâmetro de locação, \(\sigma > 0\) é
um parâmetro de escala e \(\nu >0\) é o parâmetro chamado graus de
liberdade. Uma representação estocástica, que pode facilitar o uso dessa
distribuição, é dada por

\[
Y = \mu + \sigma\frac{V}{\sqrt{U}}
\]

onde \(V \sim \mathcal{N}(0,\,1)\,\) e
\(U \sim \mathcal{Gama}(\frac{\nu}{2},\,\frac{\nu}{2})\,\), com \(U\) e
\(V\) independentes.

Seja \(\pmb{\theta} = (\mu,\,\sigma^{2},\,\nu)\) o vetor de parâmetros.
E a distribuição a priori para \(\pmb{\theta}\) é tal que:

\[\mathcal{p}(\pmb{\theta}) = \mathcal{p}(\mu)\mathcal{p}(\sigma^{2})\mathcal{p}(\nu)\]

onde \(\mu \sim \mathcal{N}(a,\,b)\,\),
\(\sigma^{2} \sim \mathcal{GI}(c,\,d)\,\) e
\(\nu \sim \mathcal{Gama}(e,\,f)\,\)

Simulando uma amostra de \(\pmb{Y} = (Y_1,\,Y_2,\,\ldots,\,Y_n)\) do
modelo em questão considerando \(n=1000\).

\hypertarget{opuxe7uxe3o-1-considerandoo-paruxe2metro-nu-fixo-usando-amostrador-de-gibbs}{%
\subsubsection{\texorpdfstring{Opção 1: Considerandoo parâmetro \(\nu\)
fixo (usando amostrador de
Gibbs)}{Opção 1: Considerandoo parâmetro \textbackslash nu fixo (usando amostrador de Gibbs)}}\label{opuxe7uxe3o-1-considerandoo-paruxe2metro-nu-fixo-usando-amostrador-de-gibbs}}

\texttt{Sugestão:\ Simule\ os\ dados\ considerando\ \$\textbackslash{}nu\ \textgreater{}\ 2\$.}

\begin{center}\rule{0.5\linewidth}{\linethickness}\end{center}

Resolução

Seja \(\pmb{\theta} = (\mu,\,\sigma^{2},\,\nu)\)

\[
\begin{align}
\mathcal{P}(\pmb{Y}|\pmb{\theta}) 
& = \prod_{i=1}^{n}\mathcal{P}(Y_i|\pmb{\theta}) \\
& =
\prod_{i=1}^{n}\mathcal{P}(Y_i|\mu,\,\sigma,\,\nu) \\
& =
\prod_{i=1}^{n}\left[\frac{\Gamma(\frac{\nu+1}{2})} {\sqrt{\nu\pi}\,\Gamma(\frac{\nu}{2})} \left(1+\frac{1}{\nu}\left(\frac{y_i-\mu}{\sigma}\right)^{2} \right)^{\!-\frac{\nu+1}{2}}\right] \\
\end{align}
\] Logo, não é trivial evoluir utilizando a distribuição de t-Student.

Daí, utilizando a representação estocástica
\(Y = \mu + \sigma\frac{V}{\sqrt{U}}\) temos que:

\[
\begin{align}
\mathcal{P}(\pmb{Y_i}|\pmb{\theta}) = \mathcal{P}(\pmb{Y_i}|\mu,\,\sigma^{2},\,U_i),\, \text{ onde }\nu \text{ é fixado e }\,V\sim\,N(0,1).
\end{align}
\] E, portanto,
\[Yi|\mu,\,\sigma^{2},\,U_i\sim\,Normal(\mu,\frac{\sigma^2}{U_1})
\]

Agora vamos encontrar o núcleo da distrib. a posteriori e a condicional
completa.

Encontrando o núcleo da distrib. a posteriori

\[
\begin{align}
& \mathcal{P}(\mu,\,\sigma^{2},\,U_i|Y_i) \propto \mathcal{P}(\pmb{Y_i}|\mu,\,\sigma^{2},\,U_i)\mathcal{P}(\mu)\mathcal{P}(\sigma^2)\mathcal{P}(U_i) \\
& \propto \underbrace{\left(2\pi\frac{\sigma^2}{U_i}\right)^{-\frac{n}{2}}exp\{-\frac{U_i}{2\sigma^2}\sum_{i=1}^{n}(y_i-\mu)^2\}}_{Normal(\mu,\frac{\sigma^2}{U_1})} \underbrace{\left(2\pi\,b\right)^{-\frac{1}{2}}exp\{-\frac{1}{2b}(\mu-a)^2\}}_{\mathcal{N}(a,\,b)} \\
& \underbrace{\frac{d^c}{\Gamma(c)}(\sigma^2)^{-(c+1)}e^{\frac{-d}{\sigma^2}}}_{\mathcal{GI}(c,\,d)} \underbrace{\frac{\left(\frac{\nu}{2}\right)^{\frac{\nu}{2}}}{\Gamma(\frac{\nu}{2})}U_i^{\frac{\nu}{2}-1}exp\{-\frac{\nu}{2}U_i\}}_{\mathcal{Gama}(\frac{\nu}{2},\,\frac{\nu}{2})}
\end{align}
\]

\$\$ \begin{align}
\mathcal{P}(\pmb{Y}|\pmb{\theta}) 
& = \prod_{i=1}^{n}\mathcal{P}(Y_i|\pmb{\theta}) \\
& =

\end{align} \$\$

\begin{Shaded}
\begin{Highlighting}[]
\CommentTok{# Fixando os valores:}
\NormalTok{mu.real<-}\StringTok{ }\DecValTok{5}
\NormalTok{sigma.real<-}\FloatTok{2.27}
\NormalTok{nu.real<-}\StringTok{ }\DecValTok{7}

\CommentTok{# Gerando os dados:}
\NormalTok{dados<-}\OtherTok{NULL}
\NormalTok{n<-}\DecValTok{1000}
\KeywordTok{library}\NormalTok{(metRology)}
\NormalTok{dados<-}\KeywordTok{rt.scaled}\NormalTok{(}\DecValTok{1000}\NormalTok{, }\DataTypeTok{df=}\NormalTok{nu.real, }\DataTypeTok{mean=}\NormalTok{mu.real, }\DataTypeTok{sd=}\NormalTok{sigma.real)}

\CommentTok{# Plot da função de densidade e histograma dos dados}
\NormalTok{bin =}\StringTok{ }\DecValTok{50}
\KeywordTok{require}\NormalTok{(graphics)}
\KeywordTok{hist}\NormalTok{(dados,}\DataTypeTok{breaks=}\NormalTok{bin, }\DataTypeTok{probability=}\OtherTok{TRUE}\NormalTok{, }\DataTypeTok{ylim =} \KeywordTok{c}\NormalTok{(}\DecValTok{0}\NormalTok{,}\FloatTok{0.25}\NormalTok{),}
     \DataTypeTok{main =} \StringTok{"Histograma dos dados"}\NormalTok{, }\DataTypeTok{col=}\KeywordTok{rainbow}\NormalTok{(bin, }\DataTypeTok{start =} \FloatTok{.78}\NormalTok{), }\DataTypeTok{ylab =} \StringTok{"F.D.P."}\NormalTok{)}
\NormalTok{x<-}\KeywordTok{seq}\NormalTok{(}\DecValTok{0}\NormalTok{,}\DecValTok{25}\NormalTok{, }\FloatTok{0.05}\NormalTok{)}
\KeywordTok{lines}\NormalTok{(x,}\KeywordTok{dt.scaled}\NormalTok{(x,}\DataTypeTok{df =}\NormalTok{ nu.real, }\DataTypeTok{mean=}\NormalTok{mu.real,}
                  \DataTypeTok{sd=}\NormalTok{sigma.real), }\DataTypeTok{col=}\StringTok{"purple"}\NormalTok{, }\DataTypeTok{lwd =} \DecValTok{2}\NormalTok{)}
\end{Highlighting}
\end{Shaded}

\includegraphics{script_files/figure-latex/unnamed-chunk-1-1.pdf}

\begin{Shaded}
\begin{Highlighting}[]
\CommentTok{##### Algoritmo de Gibbs}
\CommentTok{# Inicializando os vetores}
\NormalTok{mu<-}\OtherTok{NULL}
\NormalTok{sigma<-}\OtherTok{NULL}
\NormalTok{nu<-}\OtherTok{NULL}
\NormalTok{ite<-}\DecValTok{10000}
\NormalTok{n<-}\KeywordTok{length}\NormalTok{(dados)}
\NormalTok{prob.aux<-}\OtherTok{NULL}

\CommentTok{# Valores iniciais}
\NormalTok{mu[}\DecValTok{1}\NormalTok{]<-}\DecValTok{1}
\NormalTok{sigma[}\DecValTok{1}\NormalTok{]<-}\DecValTok{1}
\NormalTok{nu[}\DecValTok{1}\NormalTok{]<-}\DecValTok{1}

\CommentTok{# Hiperparâmetros}
\NormalTok{a<-}\DecValTok{1}
\NormalTok{b<-}\FloatTok{0.1}
\NormalTok{c<-}\DecValTok{1}
\NormalTok{d<-}\FloatTok{0.1}  
\NormalTok{e<-}\DecValTok{1}
\NormalTok{f<-}\FloatTok{0.1}

\CommentTok{# MCMC}
\end{Highlighting}
\end{Shaded}


\end{document}
